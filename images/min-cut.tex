\documentclass{standalone}
\usepackage[top=2cm, bottom=2cm, left=2cm, right=2cm, includefoot]{geometry}
\usepackage[utf8]{inputenc}
\usepackage{import}
\usepackage{amsmath}
\usepackage{amssymb}
\usepackage{bm}
\usepackage{tikz}
\usepackage{tikz-3dplot}
\usepackage{pgfplots}
\usepackage{graphicx}
\usepackage{standalone}
\usepackage{float}
\usepackage{multirow}
\usepackage{algorithm}
\usepackage{algorithmic}
\usepackage{mathtools}
\usepackage{float}
\usepackage{subcaption}
\usepackage[justification=centering]{caption}
\usepackage{multicol}
\usepackage{xspace}
\usepackage[%
  colorlinks=true, % set to false for printing
%  ocgcolorlinks,
  linktoc=all,
  linkcolor=blue!50!black,
  citecolor=blue!50!black,
  filecolor=blue!50!black,
  urlcolor=blue!50!black,
  pdfstartview = FitH,
  bookmarksopen,
  bookmarksopenlevel = 1]%
{hyperref}
\usepackage{cleveref}

\usetikzlibrary{shapes, positioning, arrows, automata, calc}

\tikzset{
	every edge/.style={ 
		draw, ->,>=stealth, auto, semithick
	}
}

\let\stdvec\vec
\renewcommand{\vec}[1]{\ensuremath{\bm{#1}}}
\newcommand{\mat}[1]{\ensuremath{\bm{#1}}}
\newcommand{\inv}[1]{\ensuremath{\mat{#1}^{-1}}}
\newcommand{\pinv}[1]{\ensuremath{\mat{#1}^{+}}}
\newcommand{\transp}{^T}
\newcommand*\dif{\mathop{}\!\mathrm{d}}
\newcommand{\euler}{\mathrm{e}}
\newcommand{\covar}{\ensuremath{\mat{\Sigma}}}
\newcommand{\pd}[2]{\dfrac{\partial #1}{\partial #2}}
\newcommand{\pds}[2]{\dfrac{\partial^2 #1}{\partial #2^2}}
\newcommand{\pdsv}[2]{\dfrac{\partial^2 #1}{\partial #2\partial #2\transp}}
\newcommand{\ad}[2]{\dfrac{\mathrm{d}#1}{\mathrm{d}#2}}
\newcommand{\ltwo}[1]{\ensuremath{\lVert #1 \rVert_2}}
\newcommand{\lone}[1]{\ensuremath{\lVert #1 \rVert_1}}
\newcommand{\norm}[1]{\ensuremath{\lVert #1 \rVert}}
\newcommand{\mean}[1]{\ensuremath{\bar{#1}}}
\newcommand{\idx}[2]{#1_{\mathrm{#2}}}
\newcommand{\eqex}{\ensuremath{\overset{!}{=}}}
\newcommand{\define}{\ensuremath{\coloneqq}}
\DeclareRobustCommand{\rchi}{{\mathpalette\irchi\relax}}
\newcommand{\irchi}[2]{\raisebox{\depth}{$#1\chi$}}
\DeclareMathOperator*{\argmax}{arg\,max}
\DeclareMathOperator*{\argmin}{arg\,min}
\DeclareMathOperator{\sign}{sgn}
\DeclareMathOperator*{\trace}{\mathrm{tr}}
\DeclareMathOperator*{\maximise}{\mathrm{maximise}}
\DeclareMathOperator*{\minimise}{\mathrm{minimise}}
\DeclareMathOperator*{\diag}{\mathrm{diag}}
\DeclareMathOperator*{\dist}{\mathrm{dist}}
\DeclareMathOperator*{\erf}{\mathrm{erf}}
\DeclareMathOperator*{\grad}{\mathrm{grad}}
\DeclareMathOperator*{\kurt}{\mathrm{kurt}}
\DeclareMathOperator*{\mutinf}{\mathrm{MI}}
\DeclareMathOperator{\sgn}{sgn}

\newcommand*{\eg}{e.g.\@\xspace}
\newcommand*{\Eg}{E.g.\@\xspace}
\newcommand*{\ie}{i.e.\@\xspace}
\newcommand*{\wrt}{w.r.t.\@\xspace}

\makeatletter
\newcommand*{\etc}{%
    \@ifnextchar{.}%
        {etc}%
        {etc.\@\xspace}%
}
\makeatother

\begin{document}
\tdplotsetmaincoords{70}{9}
\begin{tikzpicture}[tdplot_main_coords, scale=1.25] 
	\def\sz{3}
	\def\rad{0.05cm}
	\coordinate (s) at (0.5*\sz,0.5*\sz,0.75*\sz);
	\coordinate (t) at (0.5*\sz,0.5*\sz,-0.75*\sz);
	\foreach \y in {0, 1, ..., \sz} {
		\foreach \x in {0, 1, ..., \sz}
			\draw[thin, black!60] (t) -- (\x, \y, 0);
	}
	\fill[white, opacity=0.5] (0,0,0) -- ++(\sz,0,0) -- ++(0,\sz,0) -- ++(-\sz,0,0) -- cycle;
	\foreach \x in {0, 1, ..., \sz}
		\draw (\x, 0, 0) -- ++(0, \sz, 0);
	\foreach \y in {0, 1, ..., \sz}
		\draw (0, \y, 0) -- ++(\sz, 0, 0);
		\draw[very thick, white, rounded corners=5pt] (0.5,-1.5,0) -- ++(0,2,0) -- ++(1,0,0) -- ++(0,1,0) -- ++(-1,0,0) -- ++(0,1,0) -- ++(3,0,0);
	\draw[thick, blue!70!black, rounded corners=5pt] (0.5,-1.5,0) -- ++(0,2,0) -- ++(1,0,0) -- ++(0,1,0) -- ++(-1,0,0) -- ++(0,1,0) -- ++(3,0,0) node[anchor=west, black] {cut};
	\foreach \y in {0, 1, ..., \sz} {
		\foreach \x in {0, 1, ..., \sz} {
			\draw[thin, black!60] (\x, \y, 0) -- (s);
		}
	}
	\draw (0,0,0) edge node[pos=1, align=center, anchor=east] {label 1:\\linked to the drain} ++(-0.5,-0.5,0.25);
	\draw (\sz,\sz,0) edge node[pos=1, align=center, anchor=south west] {label 0:\\linked to the source} ++(0.5,0.5,0.25);
	\foreach \y in {0, 1, ..., \sz} {
		\foreach \x in {0, 1, ..., \sz} {
			\shade[ball color=black] (\x,\y,0) circle (\rad);
		}
	}
	\draw[thin, black!60] (3, 0, 0) -- (s); % 
	\shade[ball color=black] (s) circle (\rad) node[above] {$s$};	
	\shade[ball color=black] (t) circle (\rad) node[below] {$t$};
\end{tikzpicture}
\end{document}